\documentclass{beamer}

\usepackage[utf8]{inputenc}
\usepackage{default}
\usepackage[portuguese]{babel}

\usepackage{amssymb} %Para usar o símbolo \circledR

\title{Minicurso - GNU Octave}
\author{Rafael Fagundes Rosa Campos}
\date{}

\begin{document}

 \frame{\titlepage}
 
 \frame{\tableofcontents}
  
  \section{Introdução}
   \subsection{O Que é o GNU Octave?}
    \begin{frame}{O que é o GNU Octave?}
     De acordo com a Wikipedia,
     \begin{quote}
      ``GNU Octave é uma linguagem computacional, desenvolvida para computação matemática. Possui uma interface em linha de comando para a solução de problemas numéricos, lineares e não-lineares, também é usada em experimentos numéricos.''
     \end{quote}
   \end{frame}
   
   \begin{frame}{História}
     \begin{itemize}
      \item[1988] Concepção do software para facilitar os cálculos em um curso de projeto de reatores químicos \pause
      \item[1992] John W. Eaton começa o desenvolvimento. Batizada em homenagem ao professor Octave Levenspiel, conhecido por sua habilidade em ``contas de fundo de envelope''. \pause
      \item[1994] Versão 1.0 é lançada \pause
      \item[2015] Lançamento da versão 4.0, já contando com GUI
     \end{itemize}
   \end{frame}
   
   \begin{frame}{Detalhes técnicos}
    \begin{itemize}
     \item Escrito em \textbf{C++} \pause
     \item Linguagem de programação interpretada \pause
     \item Suporta a bilioteca padrão \textbf{C} e \textbf{UNIX} \textit{system calls} \pause
     \item Disponível sob a \textbf{GNU General Public License}, roda em ambientes Linux, Windows, Mac, Android e Web(\url{https://octave-online.net/})
    \end{itemize}
   \end{frame}

   
   \subsection{Para quem é esse minicurso?}
   \begin{frame}{Para que é esse minicurso?}
    \begin{itemize}
     \item Você tem acesso a um computador, smartphone, ou qualquer coisa que se conecte à internet? \pause
     \item Você não é um profissional de TI e precisa usar programação para resolver algum problema técnico ou científico? \pause
     \item Sente calafrios toda vez que escuta a palavra "FORTRAN``? \pause
     \item Acha o MATLAB\textsuperscript{\circledR} fechado demais?
    \end{itemize}
   \end{frame}
   
   \subsection{Para quem não é esse minicurso?}
   \begin{frame}{Talvez você ache o minicurso difícil, se...}
    \begin{itemize}
     \item Você nunca somou duas matrizes na vida? \pause
     \item Tem medo da interface de linha de comando? \pause
    \end{itemize}
   \end{frame}
   
   
\end{document}
